\documentclass[a4paper]{article}

\usepackage{graphicx}
\usepackage{multirow}
\usepackage{amsmath}
\usepackage{enumitem}
\usepackage{blindtext}
\usepackage{listings}
\usepackage{tikz}
\usetikzlibrary{automata,positioning,arrows}



\usepackage{xepersian}
\settextfont{B Roya}
\setlatintextfont{Tahoma}

\title{تمرین سوم اتوماتا}
\author{نیما بهرنگ 96100114}
\date{\today}	
\begin{document}
\maketitle
\centering{استاد خزایی}


\section*{پرسش ۱}

\begin{enumerate}

\item{}

\begin{latin}
$ G = (V,T,P,S), V=\{S\}, T=\{0,1\}, P = \{S \to SS | 1S0 | 0S1 | \epsilon\} $
\end{latin}
چون تعداد ۱ ها با صفرها برابر است پس اگر از ابتدا شروع کنیم ابتدا تعداد هر دو صفر و اختلافشان صفر است و با ادامه پیمایش اگر اختلاف بین آنها ناصفر باشد، می دانیم در انتها نیز اختلاف صفر است پس حداقل در نقطه ای از مسیر باید اختلاف صفر شود.\\
پس از ابتدا تا آن نقطه و از آن نقطه تا انتها خود دو کلمه هستند که در گرامر ما هستند پس می توان کلمه را به حالت کوچکتر شکاند.\\
اگر ابتدا با صفر شروع کنیم،اولین رخداد برابری با یک تمام می شود زیرا همواره تعداد صفر ها بوده پس تمام کردن با صفر به معنی زیاد بودن یک هاست که تناقض است.\\
پس کلمه به کلمه ای که با صفر شروع و با یک پایان می یابند، می شکند و به صورت بازگشتی حل می شود.

\item{}

\begin{latin}

$ \delta( q_0,0,Z_0) = \{( q_1,0Z_0 )\} $\\
$ \delta( q_0,1,Z_0) = \{( q_1,1Z_0 )\} $\\
$ \delta( q_1,0,0) = \{( q_1,00 )\} $\\
$ \delta( q_1,0,1) = \{( q_1,\epsilon)\} $\\
$ \delta( q_1,1,0) = \{( q_1,\epsilon)\} $\\
$ \delta( q_1,1,1) = \{( q_1,111)\} $\\
$ \delta( q_1,\epsilon,Z_0) = \{( q_0,Z_0)\} $\\
$q_0$ is an accept
\end{latin}

طبق تعریف ماشین پشته ای قطعی، این توصیف یک ماشین قطعی پشته ای است.\\
در ابتدا با هر کدام از ۰ یا ۱ می توان به استیت بعدی رفت و بار هر بار خواندن ۰ یا ۱ اگر در حالت قبل نیز همانند آن خوانده بودیم، همان عدد را به استک اضافه می کنیم\\
اگر چیزی که الان خواندین با قیلی فرق داشت، آن حرف را از استک فقط حذف می کنیم\\
با این کار به ازای هر ۱ یک ۰ وجود دارد و بلعکس، زیرا هر ۱ یا ایک ۰ را از استک حذف کرده و یا یک ۱ به استک اضافه و برای ۰ نیز به طور مشابه.\\
پس در انتها همواره تعداد صفر و یک برابر است.\\
نکته: در استک به غیر از 
\lr{$Z_0$}
همواره یا تعدادی یک داریم یا صفر. زیرا اگر کمی صفر در انتها باشد و بخواهد یک اضافه شود، طبق تعریف باید یک صفر حذف کند و کاری نکند. به طور مشابه برای ۱.\\
حال هر رشته ای که تعداد صفر و یک آن برابر باشد یا با یک شروع می شود یا با صفر که متناظر رفتن به استیت
\lr{$q_1$}
است و ادر  یک ها برابرند پس در انتها استک خالی می شود زیرا فقط می تواند ۰ یا ۱ در استک باشد که به دلیل برابری صفر و یک ها،چیزی نمی تواند باشد.\\
پس هر کلمه ای را هم می سازیم.
\end{enumerate}

\pagebreak
\section*{پرسش ۲}


\pagebreak
\section*{پرسش ۳}
\begin{enumerate}
\item{}
\lr{$A \rightarrow BC$}\\
\lr{$B \rightarrow \epsilon$}\\
\lr{$C \rightarrow a$}\\
این گرامر خطی نیست چون دو متغییر در سمت راست دارد.
وقطعی است زیرا توسط ماشین زیر پذیرفته می شود\\
\begin{latin}
$ \delta(q_0,a,z_0) = \{(q_1,z_0)\}$\\
$ q_1 $ is final
\end{latin}

\item{}
می دانیم که پالیندروم ها توسط پشته ای قطعی نمی توانند تولید شوند پس ناقطعی اند.
گرامر زیر نیز پالیندروم روی
\lr{a,b}
را می سازد و خطی و نامبهم است زیرا تنها یک روش برای تولید درخت است آن هم اینکه حروف را از چپ یکی یکی بسازیم\\
\begin{latin}
$ S \rightarrow aSa|bSb|\epsilon$
\end{latin}

\end{enumerate}
\pagebreak

\section*{پرسش ۴}
\begin{enumerate}[label=\Alph*]
\item{)}
\lr{$a^*ba^*ba^*$}
چون کلمه در ابتدا تبدیل به دو
\lr{T}
می شود و هر
\lr{T}
خاصیت زیر را دارد که به صورت تعدادی 
\lr{a}
در سمت چپ که می تواند تعدادش صفر باشد و تعدادی 
\lr{a}
در سمت راست که باز می تواند صفر باشد و یک 
\lr{b}
است.\\
پس دوتا از این ها پشت هم می شود تعدادی
\lr{a}
سپس
\lr{b,a}
سپش تعدادی 
\lr{a}
و سپش
\lr{b}
و در آخر هم تعدادی
\lr{a}

\item{}
این گرامر نشان دهنده یک کلمه پالیندروم است که تنها در یک کاراکتر دارای اشکال است.\\
\lr{S}
هر بار که با قانون یک و دو تبدیل به خودش شود یک 
\lr{b}
یا 
\lr{a}
به ابتدا و انتهای کلمه می افزاید.\\
هنگامی که به قوانین ۳ یا ۴ برود جایی است که نقطه خطای پالیندروم بودن است و از اینجا به بعد در مرکز این رشته فقط یک متغیر 
\lr{A}
داریم که 
\lr{A}
دوباره یا به صورت متقارن عمل می کند یا عضو وسطی رشته پالیندروم که می تواند نباشد ویا 
\lr{b,a}
باشد را به جای خود می گذارد وتمام می شود.
\item{}
این گرامر شامل همه رشته های شامل
\lr{b,a}
است.\\
به اشتقاق های زیر دقت کنید\\
\begin{latin}
$S \rightarrow ST \rightarrow Sb $ \\
$S \rightarrow ST \rightarrow SaS \rightarrow Sa$\\
$S \rightarrow Sb | Sa | \epsilon$\\
\end{latin}
این نشان می دهد که با استفاده از 
\lr{S}
می تواند شروع کرد و از راست تک تک حروف یک رشته شامل
\lr{a,b}
را ساخت
و در انتها 
\lr{S}
به اپسیلون می رود.
\end{enumerate}

\pagebreak

\section*{پرسش ۵}
\begin{enumerate}[label=\Alph*]
\item{)}
\begin{latin}
$S \rightarrow aSb|aSa|bSa|bSb|a$\\
\end{latin}
این گرامر به صورت زوج زوج تمام حالات ممکن از حروف 
\lr{a,b}
را می سازد و طولش همواره زوج می ماند و در آخر حرف
\lr{a}
را در وسط می گذارد و طولش فرد می شود

\item{}
\begin{latin}

$T\rightarrow aSa|bYb$\\
$S \rightarrow aSb|aSa|bSa|bSb|a$\\
$Y \rightarrow aYb|aYa|bYa|bYb|b$\\
\end{latin}

این نیز همانند قبلی ولی با این تفاوت که اگر حرف اول و آخر 
\lr{a}
باشد به 
\lr{S}
می رود و می شود همانند مسئله قبلی و اگر حرف اول و آخرش 
\lr{b}
باشد می رود به
\lr{Y}
که همانند قبلی است ولی با این تفاوت که حرف وسطش می شود
\lr{b}

\end{enumerate}
\pagebreak

\section*{پرسش ۶}
\begin{enumerate}
\item{}
دو اشتقاق چپترین برای
\lr{aba}
ارائه می کنیم\\
\lr{$ S \rightarrow SS  \rightarrow aS \rightarrow aSS \rightarrow abS \rightarrow aba$}\\
\lr{$ S \rightarrow SS  \rightarrow SSS \rightarrow aSS \rightarrow abS \rightarrow aba$}\\

گرامر پیشنهادی:
\lr{$ S \rightarrow aS|bS|a|b$}\\
زبان قبلی زبان تمام رشته های شامل
\lr{a,b}
بود و این گرامر نیز هست و به این صورت که از سمت چپ یکی یکی حروف رشته را می سازد تا به آخرین حرف برسد که
\lr{a,b}
است.\\
ابهام ندارد زیرا درخت آن معادل اشتقاق چپترین و اشتقاق چپ ترینش نیز تنها به صورت حرف به حرف از سمت چپ جلو می رود و یک حالت بیشتر ندارد.

\end{enumerate}

\pagebreak

\section*{پرسش ۷}

\pagebreak
\section*{پرسش۸}
\begin{latin}
$ S \rightarrow BC $\\
$ B \rightarrow SQ $\\
$ C \rightarrow SE $\\
$ E \rightarrow ) $\\
$ Q \rightarrow ( $\\

\end{latin}
\pagebreak
\section*{پرسش۹}
\pagebreak
\section*{پرسش۱۰}
یک 
\lr{nfa}
می سازیم که ماشین پشته ای خواسته شده را توصیف کند.\\
چون حافظه حداکثر 
\lr{k}
است پس تعداد حالات ممکن استک
\lr{$ \Gamma^k = p$}
یعنی تناظر بین هر حالت استک و یک عدد بین ۱ تا 
\lr{p}
است.
ماشین ما به این صورت است که به ازای هر استیت ماشین پشته ای 
\lr{p}
استیت می سازیم.\\
در واقع ماشین ما استیت هایش برابر:
\lr{$p \times Q$}
است و شروع به ساحت ماشین غیرقطعی محدودمان می کنیم به این صورت که در هر استیت از ماشین پشته ای با یک وضعیت خاصی از استک به وضعیت جدیدی از استک و استیتی می رویم که طبق توصیف آنی آن می توان مشخص کرد که از چه استیت و کدام وضعیت استک به کجا باید رفت.\\
پس به طور متناظر باری این ماشین پشته ای، یک ماشین محدود قطعی وجود دارد که نشان دهنده منظم بودن آن است
\pagebreak
\section*{پرسش۱۱}
\pagebreak
\section*{پرسش۱۱}
\pagebreak
\section*{پرسش۱۲}
\begin{enumerate}

\item{}
می دانیم که 
\lr{$2^n > n : n \geq 0$}
حال طبق لم پامپینگ اگر یک\\
\lr{n, uvwxy, |vx| > 0, |uvx| < n+1}\\
داشته باشیم و کلمه
\lr{$a^{2^n}$}
 هر
\lr{$uv^iwx^iy$}
هم عضو زبان باشد، پس به ازای افزودن 
\lr{i}
به تعداد
\lr{a}
ها افزوده می شود و این مقدار کمتر از
\lr{n}
است پس کلمه حاصله اندازه اش از 
\lr{$2^{n+1}$}
کمتر است زیرا
\lr{$2^{n+1} = 2^n + 2^n > 2^n + n \geq 2^n + i : n \geq i$}

\item{}
باز طبق لم پامپینگ جلو می رویم.
\lr{$a^nb^{2n}c^n$}\\
چون
\lr{$|vwx| \leq n$}
است، پس  
\lr{vwx}
همواره شامل یکی از 
\lr{a or c}
نیست.
زیرا باید کل 
\lr{b}
را شامل شود که نتیجه می دهد طولش از 
\lr{2n}
بیشتر است.
پس با افزودن یا کاهش
\lr{i}
،تعداد باقی حروف تغییر می کنید و حرفی که شاملش نمی شد یعنی
\lr{a or c}
تعدادش ثابت است که این تناقض در فرم رشته ها است که باید به شکل
\lr{$a^nb^2nc^n$}
باشد

\item{}
باز طبق لم پامپینگ جلو می رویم.
\lr{$b^{2n+1}c^na^{2n+1}$}\\
همانند قسمت قبل همواره شامل حداکثر یکی از 
\lr{a or b}
است.
پس ۵ حالت برای رشته 
\lr{vx}
 داریم\\
فقط شامل
\lr{c}
که با زیاد کردن 
\lr{i}
به اندازه
\lr{2n+1}
دیگر تعداد
\lr{a}
برابر بیشینه 
\lr{c,b}
که حداقل می شود
\lr{3n+1}
نیست که یک تناقض است\\
برای فقط شامل 
\lr{a}
یا فقط شامل
\lr{b}
نیز به طور مشابه صادق است\\
اگر شامل 
\lr{c,a}
باشد ، بازیاد کردن
\lr{i}
 به اندازه یک واحد، تعداد
 \lr{c}
 ها حداکثر
 \lr{n}
 تا زیاد می شود که باز کمتر از
 \lr{b}
 است ولی تعداد a
 از ماکسیمم بیشتر است که در تضاد با فرض مسئله است.
\\
اگر هم شامل
\lr{c,b}
باشد.با زیاد کردن
\lr{i}
تعداد 
\lr{b,c}
هر دو زیاد می شود اما تعداد
\lr{a}
ثابت می ماند و دیگر برابر ماکسیمم نیست که تناقض است.




\end{enumerate}

\pagebreak
\section*{پرسش۱۳}
طبق مسئله ای که در تمرین های قبل داشتیم، اگر زبانی به صورت اجتماع تعداد متناهی تصاعد حسابی باشد، منظم است.\\
این زبان چون گرامر مستقل از متن است، پس طبق لم تزریق 
برای این زبان یک عدد
\lr{n}
وجود دارد که خواص لم پامپینگ را دارد.
تمام کلمه های زبان که کوچکتر از
\lr{ n}
 کاراکتر باشد، منظم اند زیرا محدودند.\\
حال کلمات بزرگ تر از
\lr{ n}.\\
اگر طول این کلمات را باقی مانده به 
\lr{ n}
بگیریم، 
\lr{ n}
حالت دارند و اگر لم پامپینگ را رویشان اعمال کنیم اندازه
\lr{vx}
تع عددی بین ۱ تا 
 \lr{ n}
 است که آنرا عدد تصاعدی می نامیم.\\
با تعداد حداکثر
\lr{$ n \times n $}
تصاعد می توان تمام کلمات بزرگتر از 
\lr{ n}
راساخت
زیرا 
\lr{ n}
حالت برای طول کلمه به هنگ 
\lr{ n}
داریم و همچنین
\lr{ n}
حالت برای عدد تصاعدی آن.
و اگر کوچکترین اعداد با ای ویژگی ها را بگیریم، طبق لم تزریق ، تصاعدی می سازند که تمام باقی اعداد را می سازد.
پس به تعداد محدودی تصاعد داریم که نتیجه می دهد این کلمات هم منظم هستند و زبان کل می شود اجتماع دو زبان منظم که منظم است.
\pagebreak
\section*{پرسش۱۴}
هر گرامر یک ماشین پشته ای متناظر دارد. آن را
\lr{P}
می نامیم.
\\
\lr{$P^\prime$}
را اینگونه می سازیم که متشکل از دو 
\lr{P}
 است که در اولی دقیقا همان است و دومی به این صورت است که به ازای هر یال،به جای خواندن یک حرف، همواره اپسیلون خوانده می شود یعنی
\lr{ $(p,a,X) \rightarrow (p,\epsilon,X)$ }\\
حال از هر استیت اولی به استیت متناظرش در پایین یک یال \\به صورت
\lr{$ \epsilon,X,X : X $ is any character of $\Gamma$}\\
ماشینی که کپی شده کامل از ماشین اصلی است را ماشین بالایی و ماشینی که کمی تغییر کرده،ماشین پایینی می نامیم.\\
حال این ماشین هر پیشوندی را می پذیرد زیرا کافی است مقداری را در اولی برود و سپس ادامه را به پایینی بیاید و با اپسیلون پیمایش کند.
هر چیزی هم که ماشین بپذیرد، مقداری را در اولی بوده که اگر همانجا به فاینال برسد یعنی کلمه به صورت کامل است و اگر به پایین بیاید و به فاینال برسد، چون همان مسیری که در پایین طی کرده و به فاینال رسیده در بالا هم وجود دارد پس چیزی که در انتها ساخته کلمه ای بوده که اگر با بالایی می رفته یک کلمه زبان بوده و تنها فرقش این است که کلمات آن از یک جا به بعد با اپسیلون ساخته شده اند که تعریف پیشوند کلمه است.\\
اگر بخواهیم پیشوند اکید باشد و خود کلمه را نپذیرد کافی است فاینال های بالا را حذف کنیم
\pagebreak
\section*{پرسش۱۵}

\end{document}