\documentclass[a4paper]{article}

\usepackage{graphicx}
\usepackage{multirow}
\usepackage{amsmath}
\usepackage{enumitem}
\usepackage{blindtext}
\usepackage{listings}


\usepackage{xepersian}
\settextfont{B Roya}
\setlatintextfont{Tahoma}
\setlatintextfont{Tahoma}



\title{ تمرین اصول طراحی کامپیوتر}
\author{نیما بهرنگ 96100114}
\date{\today}

\begin{document}
\maketitle
\centering{استاد پارسا}

\begin{enumerate}
\item{تمرین}
\begin{enumerate}
\item{سوال 2.10}
\\
\begin{latin}
\begin{flushleft}
\lr{t0 = a + 4;}\\
\lr{t1 = a + 0;}\\
\lr{a[1] = t1;}\\
\lr{t0 = a[1];}\\
\lr{f = t1 + t0;}\\
\end{flushleft}
\end{latin}

\item{سوال 2.12.6}
  بله چون جمع 8 و
\lr{D}
بیشتر از 16 می شود
از اندازه بیشتر می شود و 
\lr{overflow} 
می شود
\end{enumerate}
\item{سوال 2.17}
\begin{latin}
type = I-type\\
instruction = load word : lw\\
lw \$v0, 4(\$at)
\end{latin}
توضیح:\\
\begin{latin}
rs = 1 => \$at\\
rt = 2 => \$v0\\
constant = 0x4 => 4\\
\end{latin}

نمایش باینری : \\
\begin{flushleft}
\lr{op : 100011}\\
\lr{rs : 00001}\\
\lr{rt : 00010}\\
\lr{constant : 0000000000000100}
\end{flushleft}

\item{سوال 2.21}

به سادگی با دستور
\lr{nor}
قابل پیاده سازیست

\setLR
\lr{nor \$t1, \$t2, \$zero}
\setRL

\item{سوال 2.23}\\
دستور
\lr{slt}:
اگر رجیستر دوم از سوم کمتر باشد رجیستر اول را یک می گذارد.پس مقدار 
\lr{\$t2}
یک می شود و دستور دوم صدق می کند زیرا با صفر برابر نیست.پس به 
\lr{else}
می رود و مقدارش دوتا اضافه می شود.پس 
\lr{\$t2}
برابر 3 می شود	

\item{سوال 2.38}\\

نمایش 4 بایت خانه ای که 
\lr{\$t1}
به آن اشاره می کند به صورت زیر است//
\lr{11 | 22 | 33 | 44}\\
و دستور
\lr{lbu}
مقدار 11 را در
\lr{\$t0}
ذخیره کرده و دستوری بعد مقدار آن را به خانه ای که
\lr{\$t2}
اشاره می کند می ریزد
پس مقدار خانه ای که 
\lr{\$t2}
اشاره می کند برابر
\lr{0x00000011}
است

\end{enumerate}
\end{document}


