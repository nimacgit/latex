\documentclass[a4paper]{article}

\usepackage{graphicx}
\usepackage{multirow}
\usepackage{amsmath}
\usepackage{enumitem}
\usepackage{blindtext}
\usepackage{listings}
\usepackage{tikz}
\usetikzlibrary{automata,positioning,arrows}



\usepackage{xepersian}
\settextfont{B Roya}
\setlatintextfont{Tahoma}

\title{تمرین اول احتمال}
\author{نیما بهرنگ 96100114}
\date{\today}	
\begin{document}
\maketitle
\centering{استاد خزایی}


\section*{پرسش ۱}
طبق قانون احتمال کل عمل می کنیم
\begin{enumerate}
\begin{latin}
\item{}
A: choose from first bag\\
B: choose from second bag\\
$P(Red)=P(Red|A)P(A) + P(Red|B)P(B) = \dfrac{3}{10}\times\dfrac{1}{2} + \dfrac{6}{10}\times\dfrac{1}{2}=\dfrac{9}{20}$

\item{}
$P(B|Red) = \dfrac{P(Red|B)\times P(B)}{P(Red)} = \dfrac{\dfrac{6}{10} \times \dfrac{1}{2}}{\dfrac{9}{20}} = \dfrac{2}{3}$
\end{latin}
\end{enumerate}
\pagebreak
\section*{پرسش ۲}
\begin{latin}
C: have cancer\\
T: system says he have cancer\\
$P(C|T) = \dfrac{P(T|C) \times P(C)}{P(T)} = \dfrac{P(T|C) \times P(C)}{P(T|C)P(C)+P(T|C^c)P(C^c)} =$\\
$ \dfrac{0.99 \times 0.07}{0.99\times 0.07+0.05\times 0.93}=\dfrac{59}{100}$
\end{latin}
\pagebreak
\section*{پرسش ۳}
 خط بیاید
\lr{T:}\\
سکه شماره 
\lr{i}
انتخاب شده باشد
\lr{$A_i$:}\\
\begin{latin}
$ P(A_k|T) = \dfrac{P(T|A_k) \times P(A_k)}{P(T)} =$
$ \dfrac{P(T|A_k) \times P(A_k)}{P(T|A_1)P(A_1) + ... + P(T|A_n) P(A_n)} = $\\
$ \dfrac{\dfrac{k}{n} \times \dfrac{1}{n}}{\Sigma (\dfrac{i}{n} \times \dfrac{1}{n}) } = \dfrac{2k}{n \times (n+1)} $
\end{latin}
\pagebreak

\section*{پرسش ۴}
\begin{latin}
$ C = {4,5,6}, C^c = {1,2,3}, A={1,2,4,5}, B={2,6} $\\
$ P(B|C) = \dfrac{1}{3}, P(B|C^c)= \dfrac{1}{3}, P(A|C) = \dfrac{2}{3}, P(A|C^c)=\dfrac{2}{3}$\\
$ P(AB|C) = 0, P(AB|C^c) = \dfrac{1}{3} => P(AB|C^c) > P(AB|C)$
\end{latin}
\pagebreak

\section*{پرسش ۵}
اگر بتوان فرض کرد که تعداد مراحل و عملیات ها برای اینکار اهمیت زیادی نداشته باشد می توان دوبار سکه انداخت و اگر اولی شیر و دومی خط بود، نفر اول و اگر اولی خط و دومی شیر بود، نفر دوم را انتخاب کنیم.اگر حالاتی غیر از این ها بود، مجدد دو سکه می اندازیم.\\
 احتمال انتخاب برای هردو نفر برابر
\lr{$p \times (1-p)$}
است ولی مشکل این است که ممکن است هیچ کدام از این حالات رخ ندد و مجبود به انجام مجدد شویم ولی احتمال آن همواره کوچک و کوچک تر می شود که فقط شاهد حالات دو خط و دو شیر باشیم
\begin{latin}

\end{latin}

\pagebreak

\section*{پرسش ۶}
با توجه به ابهام در صورت سوال که توپ های یک کیسه هنگامی که متفاوت رنگ هستند، ترتیب دارند یا خبر دو جواب برای مسئله داریم که حالت اول ترتیب مهم است و حالت دوم هر کیسه ۳ حالت هم احتمال دارد:سبز سبز، قرمز فرمز، سبز قرمز که هر کدام یک سوم احتمال دارند.\\
چون به صورت تصادفی توپ ها را پخش می کنیم پس احتمال ها همشانس هستند پس کافیست تعداد حالاتی که به تعداد دلخواه کیسه متفاوت رنگ دارد را تقسیم بر کل حالات کنیم.
\begin{latin}
$\Omega = $choosing 10 position from 20 and put greens on that 10 position and rest of 10, will be red. so $\dfrac{20!}{10!10!}$\\
\end{latin}
حالا 
\lr{k}
کیسه را انتخاب و در آن یک سبز و یک قرمز می گذاریم و از باقی کیسه ها نیز نصفی را انتخاب می کنیم تا قرمز و نصف دیگر سبز باشند.کیسه هایی که دو رنگ دارند، هر کدام به ۲ حالت می توانند سبز و قرمز را کنار \\هم گذاشت پس
\begin{latin}
$\dfrac{2^k \times n!}{k!(n-k)!} \times \dfrac{ (n-k)!}{\dfrac{n-k}{2}! \times \dfrac{n-k}{2}!}$
\end{latin}
جواب دوم\\
همانند قبل 
\lr{k}
تا را اتخاب و متفاوت رن می گذاریم و باقی را هم رنگ\\
\begin{latin}
$\dfrac{n! \times \dfrac{1}{3}^k}{k!(n-k)!} \times \dfrac{ (n-k)! }{\dfrac{n-k}{2}! \times \dfrac{n-k}{2}!}\times \dfrac{1}{3}^{n-k}$
\end{latin}
\pagebreak

\section*{پرسش ۷}
ابتدا برای دوتا نشان می دهیم و سپس با استدلالی مشابه برای تعداد بیشتر ثابت می کنیم
\begin{latin}
$ if A \subset B => P(A) \leq B, \forall A: A \subseteq \Omega => \forall A: P(A) \leq 1 $\\
$ A \cup B = A/B + B  => 1 \geq P(A \cup B) = P(A/B) + P(B) =>$\\
$ 1 + P(AB) \geq P(A/B) + P(AB) + P(B) = P(A) + P(B) => $\\
$P(AB) \geq P(A) + P(B) - 1 $\\
now for $E_1,... E_n$ we put $A = E_1, B = E_2E_3...E_n$ then we remove $E_1$ and put $A=E_2, B=E_3...E_n$ and so on\\
$P(E_1...E_n) \geq P(E_1) + P(E_2...E_n) - 1 \geq P(E_1) + P(E_2) + P(E_3...E_n) - 2 \geq ... \geq P(E_1) + ... + P(E_n) - (n-1)$\\
\end{latin}
\pagebreak
\section*{پرسش۸}
\begin{latin}
$A_i :$ probability of getting from point A + i to point A\\
$B_i :$ probability of getting from point A + i to point B\\
lest get B-A = n
so $A_0 = 1, A_n = 0, B_0 = 0, B_n = 1$\\
and also $A_i = 1 - B_i$ \\
$ A_i = \dfrac{1}{2}A_{i-1} + \dfrac{1}{2}A_{i+1}$, also for B\\
so $B_1 = \dfrac{1}{2}B_2 and B_2 = \dfrac{2}{3}B_3 and so on$\\
so $B_k = \dfrac{k}{k+1}B_{k+1}$
and $B_0 = 0, B_n = 1 => B_k = \dfrac{k}{n} => A_k = \dfrac{n - k}{n}$\\
k is distance from A: a
n is sum of a and b so $B_a = \dfrac{a}{a+b}$

\end{latin}

\pagebreak
\section*{پرسش۹}
\begin{latin}
we prove that for any n we always have P(white) = $\dfrac{a}{a+b}$ P(black) = $ \dfrac{b}{a+b}$\\
induction on n:\\
n = 1: P(white) = $\dfrac{a}{a+b}$, P(black) = $\dfrac{b}{a+b}$\\
n = k-1: true then:\\
n = k: \\
$W_i: $ choosing white from i-th bag\\
$B_i: $ choosing black from i-th bag\\
$P(W_k) = P(W_k|W_{k-1}) P(W_{k-1}) + P(W_k|B_{k-1}) P(B_{k-1})$\\
$P(W_k) = \dfrac{a+1}{a+b+1} \times \dfrac{a}{a+b} + \dfrac{a}{a+b+1} \times \dfrac{b}{a+b} = \dfrac{a}{a+b} $\\

$P(B_k) = P(B_k|W_{k-1}) P(W_{k-1}) + P(B_k|B_{k-1}) P(B_{k-1})$\\
$P(B_k) = \dfrac{b}{a+b+1} \times \dfrac{a}{a+b} + \dfrac{b+1}{a+b+1} \times \dfrac{b}{a+b} = \dfrac{b}{a+b} $\\

so for every n the probability of choosing white from last bag is $ \dfrac{a}{a+b} $

\end{latin}

\pagebreak
\section*{پرسش۱۰}
\begin{latin}
\begin{enumerate}
\item{}
correct \\
$ A \searrow B => P(A|B) \leq P(A) => \dfrac{P(AB)}{P(B)} \leq P(A) => \dfrac{P(AB)}{P(A)} \leq P(B) => P(B|A) \leq P(B) => B \searrow A $
\item{}
incorrect\\
$ A = C = {1,2,3}, B = {1,4}$\\
$ P(A) = \dfrac{3}{4} \geq P(A|B) = \dfrac{1}{2}  $\\
$ P(B) = \dfrac{2}{4} \geq P(B|C) = \dfrac{1}{3}  $\\
$ P(C) = \dfrac{3}{4} \leq P(C|A) = \dfrac{1}{1}  $\\
\item{}
incorrect\\
$ A = {1,2,3},  C = {3,4,5}, B = {3,6,7,8,9}$\\
$ P(A) = \dfrac{3}{9} \geq P(A|B) = \dfrac{1}{5}  $\\
$ P(C) = \dfrac{3}{9} \geq P(C|B) = \dfrac{1}{5}  $\\
$ P(AC) = \dfrac{1}{9} \leq P(C|A) = \dfrac{1}{5}  $\\

\end{enumerate}
\end{latin}

\pagebreak
\section*{پرسش۱۱}
\begin{latin}
k: initial money\\
bankruptcy: when our money get 0\\
$A_k:$ probability of getting bankrupt when we have k money\\
$ P(A_k) = P(A_k|win)P(win) + P(A_k|loose)P(loose)$\\
$P(A_k) = P(A_{k+1})p + P(A_{k-1})(1-p)$\\
we can solve it with discrete math method as an second degree equation\\
$x = p.x^2 + (1-p) => x = \dfrac{1 \pm \sqrt{1 - 4p(1-p)}}{2p}$\\
$x = \dfrac{1 \pm |2p-1|}{2p}: p \leq \dfrac{1}{2} => x = \dfrac{1 - (1-2p)}{2p} = 1:$ always bankrupt\\
$p > \dfrac{1}{2} => x = \dfrac{1 - (2p-1)}{2p} = \dfrac{1-p}{p} => x_k = (\dfrac{1-p}{p})^k$
\end{latin}

\pagebreak
\section*{پرسش۱۲}
\begin{enumerate}
\item{}
با استقرا حدس خود را ثابت می کنیم
\begin{latin}
$\dfrac{g}{r+b+g}$\\
for one ball it's obvious.\\
for more than one ball we use induction on total number of balls.\\
$A_r: $ first ball be red
$A_g: $ first ball be green
$A_b: $ first ball be blue
$G: $ last balls be green
$P(G) = P(G|A_r)P(A_r) + P(G|A_g)P(A_g) + P(G|A_b)P(A_b)$\\
we can calculate $P(G|A_k) $ with the induction we use as the total number of balls reduced by one.\\
$P(G) = \dfrac{g}{g+r+b-1} \times \dfrac{r}{r+g+b} +  \dfrac{g-1}{g+r+b-1} \times \dfrac{g}{r+g+b} +  \dfrac{g}{g+r+b-1} \times \dfrac{b}{r+g+b} = \dfrac{g}{r+g+b}$\\
so its always same.

\end{latin}
\item{}
همانند مثال قبل از استقرا استفاده می کنیم
\begin{latin}
$\dfrac{r}{r+b+g}$\\
for one ball it's obvious.\\
for more than one ball we use induction on total number of balls.\\
$A_r: $ first ball be red
$A_g: $ first ball be green
$A_b: $ first ball be blue
$R: $ red balls finished first
$P(R) = P(R|A_r)P(A_r) + P(R|A_g)P(A_g) + P(R|A_b)P(A_b)$\\
we can calculate $P(R|A_k) $ with the induction we use as the total number of balls reduced by one.\\
$P(R) = \dfrac{r-1}{g+r+b-1} \times \dfrac{r}{r+g+b} +  \dfrac{r}{g+r+b-1} \times \dfrac{g}{r+g+b} +  \dfrac{r}{g+r+b-1} \times \dfrac{b}{r+g+b} = \dfrac{r}{r+g+b}$\\
so its always same.

\end{latin}

\end{enumerate}
\pagebreak
\end{document}