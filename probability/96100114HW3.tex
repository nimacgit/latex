\documentclass[a4paper]{article}

\usepackage{graphicx}
\usepackage{multirow}
\usepackage{amsmath}
\usepackage{enumitem}
\usepackage{amsmath}
\usepackage{blindtext}
\usepackage{listings}
\usepackage{tikz}
\usetikzlibrary{automata,positioning,arrows}



\usepackage{xepersian}
\settextfont{B Roya}
\setlatintextfont{Tahoma}

\title{تمرین سوم احتمال}
\author{نیما بهرنگ 96100114}
\date{\today}	
\begin{document}
\maketitle
\centering{استاد خزایی}

\section*{پرسش ۱}
طبق قانون احتمال کل عمل می کنیم
\begin{enumerate}
\begin{latin}
\item{}
$ p_X =\Sigma_y p_{X,Y}(x,y)$
\[
  p_X=\begin{cases}
               0.4 & 1\\
               0.1 & 2\\
0  & 3\\
0.1 & 4\\
0.4 & 5				
            \end{cases}
\]
\[
  p_Y=\begin{cases}
               0.1 & 1\\
               0.3 & 2\\
0.2  & 3\\
0.4 & 4
            \end{cases}
\]
\item{}
They are independent as $ \forall x,y:  p_{X,Y}(x,y) = p_X(x)\times p_Y(y)$\\
\item{}
$E[W] = \Sigma w.P_W(w)$\\
$E[X|Y=y]=\Sigma x.p_{X|Y}(x|y)=\Sigma x.\dfrac{p_{X,Y}(x,y)}{p_Y(y)}$\\
\[
  E[X|Y=y] =\begin{cases}
& 1\\
& 2\\
0  & 3\\
0.1 & 4\\
0.4 & 5				
            \end{cases}
\]


\end{latin}
\end{enumerate}
\pagebreak
\section*{پرسش ۲}
\begin{latin}
Probability of a color appear in our event: 1 - it doesn't appear = $1 - \dfrac{(^{60}_{20})}{(^{70}_{20})}$\\
introduce $Y_i$ a new indicator that show if color i exists in our outcome.\\
$X=\Sigma Y_i$ and all of them are equal.We have seven color so the expected value of number of colors according to the linearity of expectation is $7\times (1 - \dfrac{(^{60}_{20})}{(^{70}_{20})})$

\end{latin}
\pagebreak
\section*{پرسش ۳}
\begin{latin}
Memory less: $P_x(X>m+n | X>m) = P_x(X>n)$\\
Conditional: $P(a|b) = \dfrac{P(ab)}{P(b)}$\\
If its geometric($P_x(X=k)=(1-p)^{k-1}p$) $ \Rightarrow $\\
$P_x(X>m+n|X>n) = \dfrac{P_x(X>m+n)}{P_x(X>n)} =$
$\dfrac{(1-p)^{n+m}}{(1-p)^n} = (1-p)^m = P_x(X>m)$\\

If we have a Memory-less so $P_x(X=k) = P_x(X>k) - P_x(X>k+1)=$\\
$= P_x(X>k) - P_x(X>k+1|X>1)\times P_x(X>1) = $\\
$= P_x(X>k) - P_x(X>k)\times P_x(X>1) = (1-P_x(X>1)) \times P_x(X>k)$\\
now we calculate the ratio of sequence:$\dfrac{P_x(X=k+1)}{P_x(X=k)} = \dfrac{P_x(X>k)}{P_x(X>k+1)}=\dfrac{P_x(X>k)}{P_x(X>k+1|X>1)\times P_x(X>1)} =$\\
$=\dfrac{P_x(X>k)}{P_x(X>k)\times P_x(X>1)} = Constant = \dfrac{1}{P_x(X>1)}$\\ 
So its exactly the definition of a geometric RV.


\end{latin}
\pagebreak

\section*{پرسش ۴}
\begin{latin}
$ fact1: by definition:  \Sigma_k \dfrac{(_{k}^{D})(_{(n)-(x)}^{(N)-(D)})}{(_{(n)}^{(N)})} = 1$\\
$ fact2: (_n^m) = \dfrac{m.(m-1)}{n.(n-1)} (_{n-2}^{m-2})$\\
we know that expected of a hypergeometric RV  is $ \dfrac{nD}{N} $\\
$Var(X) = E[(X-E[X])^2] = E[X^2] - E[X]^2 = E[X(X-1) + X] - E[X]^2 = E[X(X-1)] + E[X] - E[X]^2 $\\
$E[X(X-1)] = \Sigma x(x-1)P(X=x) = \Sigma x(x-1) \dfrac{(_x^D)(_{n-x}^{N-D})}{(_n^N)} = $\\
$ \Sigma_x D.(D-1) \dfrac{(_{x-2}^{D-2})(_{(n-2)-(x-2)}^{(N-2)-(D-2)})}{\dfrac{N(N-1)}{n(n-1)}\times (_{(n-2)}^{(N-2)})} = \dfrac{D(D-1).n(n-1)}{N(N-1)}$\\
$Var(X) = \dfrac{D(D-1).n(n-1)}{N(N-1)} + \dfrac{nD}{N} - (\dfrac{nD}{N})^2 = n\dfrac{D}{N}(1-\dfrac{D}{N})(1-\dfrac{n-1}{N-1})$

\end{latin}
\pagebreak

\section*{پرسش ۵}
\begin{latin}
we can make a new indicator like $Y_i$ which show card 
\end{latin}

\pagebreak

\section*{پرسش ۶}
\pagebreak

\section*{پرسش ۷}
برای اینکه آقای خسته به جایگاه برود لازم است که هر ۹ تا ماشین جلوی او رفته باشند و بعد از تعدادی ماشین، او انتخاب شود.
\begin{latin}
$\forall k>=10: P_x(k) = ( (_9^{k-1})\dfrac{1}{3}^{10}) \times (\dfrac{1}{3}^{k-10}\times 2^{k-10})$\\
$E[X] = \Sigma^{\infty}_{k=10} k.P_x(k)$
\end{latin}
\pagebreak
\section*{پرسش۸}
\begin{latin}
$E[X] = \Sigma^{\infty}_{k=1}\dfrac{i}{2^i} = \Sigma^{\infty}_{k=1}\dfrac{1}{2^i} + \Sigma^{\infty}_{k=2}\dfrac{1}{2^i} +\Sigma^{\infty}_{k=3}\dfrac{1}{2^i} + ...$\\
$= \dfrac{1}{1} + \dfrac{1}{2} + \dfrac{1}{4} + ... = 2$
\end{latin}

\pagebreak
\section*{پرسش۹}
\begin{latin}
\begin{enumerate}
\item{}
\end{enumerate}
\end{latin}

\pagebreak
\section*{پرسش۱۰}
\begin{latin}
\begin{enumerate}
\item{}
correct \\
$ A \searrow B => P(A|B) \leq P(A) => \dfrac{P(AB)}{P(B)} \leq P(A) => \dfrac{P(AB)}{P(A)} \leq P(B) => P(B|A) \leq P(B) => B \searrow A $
\item{}
incorrect\\
$ A = C = {1,2,3}, B = {1,4}$\\
$ P(A) = \dfrac{3}{4} \geq P(A|B) = \dfrac{1}{2}  $\\
$ P(B) = \dfrac{2}{4} \geq P(B|C) = \dfrac{1}{3}  $\\
$ P(C) = \dfrac{3}{4} \leq P(C|A) = \dfrac{1}{1}  $\\
\item{}
incorrect\\
$ A = {1,2,3},  C = {3,4,5}, B = {3,6,7,8,9}$\\
$ P(A) = \dfrac{3}{9} \geq P(A|B) = \dfrac{1}{5}  $\\
$ P(C) = \dfrac{3}{9} \geq P(C|B) = \dfrac{1}{5}  $\\
$ P(AC) = \dfrac{1}{9} \leq P(C|A) = \dfrac{1}{5}  $\\

\end{enumerate}
\end{latin}

\pagebreak
\section*{پرسش۱۱}
\begin{latin}
k: initial money\\
bankruptcy: when our money get 0\\
$A_k:$ probability of getting bankrupt when we have k money\\
$ P(A_k) = P(A_k|win)P(win) + P(A_k|loose)P(loose)$\\
$P(A_k) = P(A_{k+1})p + P(A_{k-1})(1-p)$\\
we can solve it with discrete math method as an second degree equation\\
$x = p.x^2 + (1-p) => x = \dfrac{1 \pm \sqrt{1 - 4p(1-p)}}{2p}$\\
$x = \dfrac{1 \pm |2p-1|}{2p}: p \leq \dfrac{1}{2} => x = \dfrac{1 - (1-2p)}{2p} = 1:$ always bankrupt\\
$p > \dfrac{1}{2} => x = \dfrac{1 - (2p-1)}{2p} = \dfrac{1-p}{p} => x_k = (\dfrac{1-p}{p})^k$
\end{latin}

\pagebreak
\section*{پرسش۱۲}
\begin{enumerate}
\item{}
با استقرا حدس خود را ثابت می کنیم
\begin{latin}
$\dfrac{g}{r+b+g}$\\
for one ball it's obvious.\\
for more than one ball we use induction on total number of balls.\\
$A_r: $ first ball be red
$A_g: $ first ball be green
$A_b: $ first ball be blue
$G: $ last balls be green
$P(G) = P(G|A_r)P(A_r) + P(G|A_g)P(A_g) + P(G|A_b)P(A_b)$\\
we can calculate $P(G|A_k) $ with the induction we use as the total number of balls reduced by one.\\
$P(G) = \dfrac{g}{g+r+b-1} \times \dfrac{r}{r+g+b} +  \dfrac{g-1}{g+r+b-1} \times \dfrac{g}{r+g+b} +  \dfrac{g}{g+r+b-1} \times \dfrac{b}{r+g+b} = \dfrac{g}{r+g+b}$\\
so its always same.

\end{latin}
\item{}
همانند مثال قبل از استقرا استفاده می کنیم
\begin{latin}
$\dfrac{r}{r+b+g}$\\
for one ball it's obvious.\\
for more than one ball we use induction on total number of balls.\\
$A_r: $ first ball be red
$A_g: $ first ball be green
$A_b: $ first ball be blue
$R: $ red balls finished first
$P(R) = P(R|A_r)P(A_r) + P(R|A_g)P(A_g) + P(R|A_b)P(A_b)$\\
we can calculate $P(R|A_k) $ with the induction we use as the total number of balls reduced by one.\\
$P(R) = \dfrac{r-1}{g+r+b-1} \times \dfrac{r}{r+g+b} +  \dfrac{r}{g+r+b-1} \times \dfrac{g}{r+g+b} +  \dfrac{r}{g+r+b-1} \times \dfrac{b}{r+g+b} = \dfrac{r}{r+g+b}$\\
so its always same.

\end{latin}

\end{enumerate}
\pagebreak
\end{document}