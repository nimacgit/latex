%بسم الله الرحمن الرحیم

%You should edit DSLecture.tex, not this file!
\usepackage{amsthm}
\usepackage{latexsym}
\usepackage{amssymb}
\usepackage{verbatim}
\usepackage{enumitem,amsmath,array}
\usepackage{tikz}
\usepackage{tkz-graph}
\usepackage{bookmark}
\usetikzlibrary{positioning,chains,fit,shapes,calc}
\usepackage[a4paper, margin=0.7in]{geometry}
\usepackage{listings}
\usepackage{clrscode3e}
\usepackage{hyperref}
\hypersetup{
	colorlinks=true,
	linkcolor=blue,
	filecolor=magenta,      
	urlcolor=cyan,
}
\usepackage[fontsloadable]{xepersian}


\settextfont{HM XNiloofar}
%\setdigitfont{HM XNiloofar}
%\setdigitfont{ParsiDigits}
\defpersianfont\outline[Scale=1]{HM XNiloofar Outline}

\setlength{\parindent}{1.5em}
\setlength{\parskip}{0.9em}
\renewcommand{\baselinestretch}{1.4}


\newcommand{\lecture}[3]{
%\pagestyle{empty}
{
	\begin{center}
			\vspace{-1cm}
		   \includegraphics[scale=0.15]{Sharif}%\hfill \\[1em]  
	\end{center}
	\vspace{-8mm}
\begin{center}

\bf
%\begin{outline} 
{
\Large
آنالیز الگوریتم‌ها (۲۲۸۹۱)
}
%\end{outline} 
\\
%مدرس: مرتضی علیمی
%\\
مدرس حل تمرین: مجید گروسی
\\~
[بهار ۹۹]
\end{center}
}\vspace*{-1em}
\noindent
جلسه حل تمرین #1: #2 \hfill نگارنده: #3
\vspace{-4mm}
\rule{\textwidth}{1pt}
\ \\
}

% example environment
\newenvironment{example}
{\smallskip \noindent \emph{مثال:}}
{\hfill $\boxtimes$ \smallskip}


\newtheorem{theorem}{قضیه}
\newtheorem{proposition}{گزاره}
\newtheorem{claim}{ادعا}
\newtheorem{lemma}{لم}
\newtheorem{corollary}{نتیجه}
\newtheorem{definition}{تعریف} % Use this for non-trivial definitions.

 %%%%%%%%%%%%%%%%%%%%%%%%%%%%%%%%%%%%%%%%%%%%%%%%%%%%%%%%%%%%%%%%%%%%%%%%%%%%

